\chapter{Bevezetés}

A digitális képfeldolgozás a 21. század egyik legszélesebb körben tért nyerő technológiája, amely az ipari gyártósorok minőség-ellenőrzésétől kezdve a forgalomszámláláson át a tömegközlekedési eszközök biztonsági felügyeletéig alakítja mindennapi életünket. A technológiának az 1960-as években letett alapjai után nagy fellendülést hozott a 2000-es évek elején megjelenő, egyre nagyobb számítási kapacitással rendelkező processzorok, valamint a 2010-es évektől egyre nagyobb figyelmet kapó gépi tanulás alapú algoritmusok.\\
\\
A diplomatervem témája erősen kötődik a számítógépes képfeldolgozáshoz, célja egy általános célú képfeldolgozó keretrendszer, a \emph{cv4s} felhasználása beágyazott környezetben. A beágyazott mikroszámítógépek jellemzően kisebb számítási teljesítménnyel rendelkeznek, mint a mindennapokban használt asztali számítógépek, ám jóval olcsóbbak, kisebbek és energiatakarékosabbak, ezáltal sokkal rugalmasabban használhatók fel a legtöbb képfeldolgozást igénylő folyamatban. \\
\\
A \emph{cv4s} egy alapvetően asztali számítógépeken történő képfeldolgozást lehetővé tévő keretrendszer, ami a \textit{Microsoft} által fejlesztett \emph{.NET Framework} révén egyrészt erősen kötődik a \emph{Windows} operációs rendszerhez, másrészt a beágyazott környezetben szokatlan menedzselt futási környezetet igényel, ami a legtöbb esetben nem kedvez a valósidejű, jellemzően nagy számítási igényű képfeldolgozási műveleteknek. Ezen okok miatt kifejezetten érdekes és kihívást jelentő feladat volt a rendszer integrációja. \\
Azért releváns célkitűzés a képfeldolgozás beágyazott rendszeren való elvégzése, mert sok alkalmazási esetben a megfigyelt objektum helye nem teszi lehetővé, vagy nem költséghatékony az asztali számítógép alkalmazása. Ilyen lehet például gyártósorok szabályozására használt képfeldolgozás, vagy a forgalom számlálására kihelyezett eszköz. Jellemzően ilyenkor az sem megoldás, hogy a nyers adatokat hálózaton keresztül küldjük feldolgozásra az adatmennyiség, vagy a megbízhatatlan hálózati kapcsolat miatt. A távoli számítógép általában csak az mikroszámítógép felügyeletére szolgál, a felhasználó távirányítással konfigurálja az eszközt, amely működése során csak a kinyert információt továbbítja.\\
\\
A diplomatervem keretében a keretrendszer már említett integrációját végeztem el, beágyazott számítógépre valamint a keretrendszert bővítettem egy valós idejű képfeldolgozást lehetővé tevő szoftver modullal, melynek tervezésekor igyekeztem a lehető legnagyobb rugalmasságot biztosítani, hogy az alkalmazás minél általánosabban felhasználható legyen egyéb felhasználási területeken is. Az elkészült szoftver hálózati kapcsolaton keresztül kommunikál a felügyelő számítógépen futó alkalmazással, amely szintén a diplomaterv keretében került megvalósításra. A felhasználó a program segítségével intuitív grafikus felület segítségével fér hozzá a beágyazott számítógépen feldolgozott adatokhoz és ezen keresztül képes az eszköz konfigurációjára is.