\chapter{Felhasznált technológiák, irodalomkutatás}
\section{Raspberry Pi/ Raspbian}

\section{Mono}

\section{Win10 IoT Core}

\section{Windows Communication Foundation}
A \emph{Windows Communication Foundation} (\emph{WCF}) a \textit{.NET Framework} 

\section{OpenCV, OpenCvSharp}

Az \emph{Open Computer Vision} (\emph{OpenCV}) egy képfeldolgozást segítő, nyílt forráskódú szoftverkönyvtár, amely kényelmes és jól dokumentált keretrendszert nyújt képfeldolgozási, valamint gépi tanulással kapcsolatos alkalmazások számára. A szoftver a BSD licensz védelme alatt érhető el, és ennek megkötésit figyelembe véve használható fel. A könyvtár eredetileg C/C++ nyelven íródott az alacsony szintű nyelvek minden optimalizációs lehetőségét kihasználva, azonban létezik hivatalos API Python nyelvhez is. \\
\\
A \emph{cv4s} keretrendszer -- lévén C\#-ban íródott -- egy harmadik féltől származó (\textit{3rd party}) burkoló szoftvercsomagon keresztül használható, amely lehetővé teszi a C/C++ nyelvről fordított binárisok használatát a C\# szintaktikájával. \\
\\
A megvalósított képfeldolgozási funkciók mindegyike az OpenCV
