\chapter{Kommunikáció} \label{chapter:kommunikacio}

A fejezet témája az elkészített rendszerrel szemben támasztott egyik legfőbb követelmény, a távoli felügyelet és konfigurálhatóságot megvalósító szoftverkomponens működése, valamint a tervezése során tett megfontolások. A rugalmasságot szem előtt tartva igyekeztem teljesen elkülöníteni a távoli vezérelhetőséget, illetve a \ref{chapter:kepfeldolgozo} fejezetben bemutatott képfeldolgozási funkciókat biztosító osztályokat.\\
Az eszköz távirányíthatóságát a .NET Framework WCF (\emph{Windows Communication Foundation}) nevű, szolgáltatás-orientált keretrendszerének segítségével oldottam meg. 


\section{Windows Communication Foundation}
Az elkészített szoftverkomponens bemutatása előtt pár szó a WCF-ről, illetve a mögötte lévő szoftvertervezési filozófiáról. A \ref{ssection:wcf_intro} számú bevezetőben leírt szolgáltatás-orientált szemlélet szerint készült C\# nyelven írt alkalmazások támogatására a Microsoft a WCF keretrendszert hozta létre. A keretrendszer menedzselt futási környezetet biztosít a szolgáltatások számára, amelyen keresztül azok CLR típusokat tehetnek közzé, illetve a fogadott adatot CLR típusként reprezentálhatják. Az alkalmazás komponensei nem igénylik a közös operációs rendszert, programozási nyelvet, de még azt sem, hogy egy számítógépen fussanak, elegendő a megfelelően konfigurált adatkapcsolat illetve a programozási felület (\emph{API}) ismerete. A WCF terminológiája a szolgáltató szoftver-komponenst \emph{service}-nek (magyarul szolgáltatásnak) nevezi, és a szolgáltatás-orientált alkalmazás több ilyen egységet aggregál, miközben ezen egységek egymástól való függőségei minimálisak. A \emph{service} ugyanis nem más mint funkcionalitások összessége, amit a komponens elérhetővé tesz az alkalmazás elemei számára, leegyszerűsítve távolról hívható függvény gyűjtemények, melyek egymástól függetlenül karban tarthatók (hasonlóan az objektumorientált szemlélet által bevezetett osztályokhoz, azonban kisebb mértékű csatolással).\\

Az így létrejövő struktúrában a szolgáltatások mellett megjelennek az őket igénybe vevő kliensek (\emph{client}), amelyek gyakorlatilag bármilyen formát ölthetnek --  Windows Forms, ASP.NET alkalmazás vagy a mi esetünkben WPF (\ref{chapter:kliens} fejezet). A \emph{service-client} architektúrában a komponenseknek egymásról csak korlátozottan van információjuk és aszinkron módon üzenetekkel kommunikálnak. A kliensek a szolgáltatások funkcionalitását és jellemzően a WCF által biztosított \emph{metaadat cserén} keresztül ismerik meg. A metaadat valamilyen technológia-független módon kerül közzétételre\footnote{Ez szinte kivétel nélkül WSDL (\emph{Web Services Description Language}) formátumú adatként történik, amely XML-ként tárolja a szükséges tulajdonságokat.}, és tartalmazza az adott szolgáltatás által nyújtott funkcionalitásokat, valamint a kommunikáció lehetséges módjait. \\

A szolgáltatás-kliens struktúrában az esetek döntő többségében csak a kliens hívhatja meg a \emph{service} metódusait, a fordított irányú kommunikáció megvalósítására a \ref{subs:rpi_log_service} alfejezetben látható példa. Általában -- és erre a megvalósított rendszernél is ki fogok térni-- a kliens nem közvetlenül a szolgáltatással kommunikál, hanem egy proxy-objektummal, amely ugyanazokkal a funkcionalitásokkal rendelkezik, mint a \emph{service} és ezen felül a kommunikációs kontextust kezelő tulajdonságai is vannak. Természetesen futási időben megismert szolgáltatás esetében nincs lehetőség a proxy-objektum definiálásra és betöltésére a memóriában, ilyenkor közvetlen kapcsolatot is létre lehet hozni a \emph{service}-szel. \cite{wcf-programming}
\begin{figure}[h]
\vspace{.5cm}
\includegraphics[scale=1.25]{wcf_abc.pdf}
\centering
\vspace{.2cm}
\caption{WCF végpont}
\vspace{.5cm}
\label{fig:wcf_abc}
\end{figure}

Egy szolgáltatásnak a következő három három összetevőt kell definiálnia az egyértelmű azonosíthatóság érdekében:
\paragraph{Address} Minden szolgáltatáshoz tartoznia kell egy egyedi címnek, amely két fontos részletében hordoz információt a kliens számára, méghozzá a meghatározza a \emph{service} helyét (a lokális számítógépen vagy a hálózaton), illetve a kommunikáció módját az alkalmazott protokoll által. A cím mindig a következő formában írható fel:

\begin{mdframed}[backgroundcolor=gray!20]
\begin{scriptsize}
\begin{lstlisting}[language=bash]
[transport]://[machine or domain][:optional port]/[optional URI]
\end{lstlisting}
\end{scriptsize}
\end{mdframed}

A WCF által támogatott, általam használt kommunikációs sémák a HTTP/HTTPS és a TCP. (Ha lokális gépen futó kliens számára szeretnénk szolgáltatást \emph{host}-olni, az IPC (\emph{inter-process communication}), MSMQ (\emph{Microsoft Message Queue}, illetve a Service Bus kommunikációs protokollok valamelyike lehet a jó megoldás.)
\paragraph{Binding} A szolgáltatással való kommunikáció több szempontból is eltérő lehet, és sokrétűen konfigurálható. A kommunikáció iránya, az üzenetek szinkronitása és a fogadás módja szerint is részletekbe menő beállításokra van szükség a kommunikációs séma teljes körű definiáláshoz. A legtöbb alkalmazásban nincs szükség minden paraméter részletekbe menő beállítására, erre a keretrendszer előre definiált megoldásokat kínál a leggyakoribb sémákra, amelyek a fentebb leírt kommunikációs protokollokra definiálnak egy-egy sémát, amelyet igény szerint tovább finomíthatunk.
\paragraph{Contract} Platformfüggetlen \emph{szerződésként} írja le a szolgáltatás képességeit a kliens számára. Az objektumorientált szemléletben használt interfészekkel állítható párhuzamba -- megvalósítása objektumorientált nyelven (esetünkben C\#) gyakorlatilag is interfészekkel történik. A WCF négy különböző \emph{contract} típust definiál, melyek a 
\begin{itemize}
\setlength\itemsep{.1em}
\item \textbf{\emph{Service contract}} Meghatározza, milyen funkcionalitásokra képes a szolgáltatás. Az osztályunk által definiált CLR interfészt összerendeli a szolgáltatás technológia-független interfészével. 
\item \textbf{\emph{Data contract}} A szolgáltatás által használt adattípusukat definiálja. A CLR \emph{érték típusai} alapértelmezetten ismertek a keretrendszer számára.
\item \textbf{\emph{Fault contract}} A \emph{service} oldali hibák kliens oldalon való jelzésére szolgáló megegyezés.
\item \textbf{\emph{Message contract}}. Szolgáltatások közvetlen, üzenet alapú információcseréjét teszi lehetővé. 
\end{itemize}
Ezen lehetőségek közül a diplomamunkámban a \emph{service contract}-ot használtam fel.\\

E három komponens együttesen egy \emph{végpontot} (\emph{endpoint}-ot) definiál a \ref{fig:wcf_abc} ábrán látható módon.
A végpontot azonosító három összetevő egyenként is számos eltérő beállítási lehetőséget nyújt, azonban összességében nézve szinte teljeskörűen az alkalmazásra szabható konfigurációt kapunk.\\
Fontos megjegyezni, hogy minden \emph{service}-hez tartoznia kell legalább egy \emph{endpoint}-nak -- de akár többnek is. A szolgáltatást futtató \emph{host} ezeket a végpontokat teszi elérhetővé a kliensek számára a már említett metaadat cserén keresztül. A kliens a metaadathoz is a szolgáltatás speciális, \emph{MEX}, azaz \emph{metadata exchange} végpontján keresztül fér hozzá, ez a WCF dokumentáció által ajánlott módszer. A végpontok konfigurációja a szolgáltatás megvalósítása során az egyik legfontosabb lépés, erre a WCF keretrendszer két módot is kínál: programkódból történő, illetve adminisztratív beállítást. A különbség a két módszer között, hogy míg az előbbi esetben a konfigurációt reprezentáló objektumokat a forráskódban explicit létrehozzuk, az utóbbi módszerrel a példányosítást a keretrendszerre bízzuk, az egyes paramétereket pedig XML formátumú konfigurációs fájlban adjuk át neki. Az utóbbi eset jelentős előnye, hogy a szoftver ebben az esetben nem függ a kommunikáció paramétereitől, a konfiguráció-változtatás pedig a gyakorlatban egyetlen fájl cseréjét jelenti. \cite{wcf-doc} \\
\\
A WCF architektúrában az üzenetek eljuttatása a klienstől a szolgáltatásig a \ref{fig:wcf_architekture} ábrán bemutatott módon, úgynevezett \code{Channel} objektumokon keresztül történik. Látható, hogy a kliens vagy egy, a szolgáltatást reprezentáló proxy-n keresztül vagy közvetlenül kommunikál a legfelső \code{Channel}-lel. Ezek az objektumok valósítják meg a beállított kommunikációs séma szükséges lépéseit, a klienstől jövő üzeneteket a transzport csatornáig eljutva a megfelelő formátumúra alakítják; a lépések struktúrája nagyrészt a \emph{Binding} konfigurációjától függ. A \emph{service} oldalon a fogadott objektumon fordított sorrendben végrehajtva a leírt transzformációkat visszakapjuk a kliens üzenetét, amit a szolgáltatás a \code{Dispatecher}-től kap meg a megfelelő végpontján. A \emph{service}-t valójában közvetlenül egy lokális kliens hívja meg -- a diszpécser. A szolgáltatás által a hívásra adott válasz a \ref{fig:wcf_architekture} ábrán látható úton ellentétes irányban jut el a klienshez. \cite{wcf-programming}

\begin{figure}[h]
\vspace{.5cm}
\includegraphics[scale=1.1]{wcf_architekture.pdf}
\centering
\vspace{.2cm}
\caption{WCF kommunikációs architektúra\cite{wcf-programming}}
\vspace{.5cm}
\label{fig:wcf_architekture}
\end{figure}

A kliens oldali modul elvárt működésének érdekében a Visual Studio-ban \code{Service Reference} hozzáadása szükséges a projektünkhöz Itt a szolgáltatás elérési címének megadása után megjelenik a kívánt \emph{service}, és a VisualStudio automatikusan legenerálja a metaadat alapján a \ref{fig:wcf_architekture} ábrán látható kliens oldali kódot a szolgáltatás használatához. Ha nem Visual Studio-t használunk lehetőség van az \code{svcutil} parancssori program alkalmazására, ami elvégzi a szolgáltatásunk kliens oldali kódjának létrehozását.

A diplomaterv keretében megvalósított alkalmazás esetén a Raspberry Pi-n futó képfeldolgozó program látja el a  \emph{service} szerepét, a folyamatot felügyelő, asztali gépen futó grafikus program pedig kliensként veszi igénybe annak szolgáltatásait, amelyeket a \ref{section:servicek} alfejezet részletez.

\section{Kommunikációs protokoll}

A keretrendszer megismerése után, az implementálás előtti utolsó lépés a kommunikációs protokoll kiválasztása volt. A WCF által támogatott hálózati protokollok közül a TCP és a HTTP közül mindkettő jó választás lehet, a weben HTTP szélesebb körben támogatott, alkalmazásrétegbeli, a TCP pedig ipari alkalmazásokhoz jobban illeszkedő, szállítási rétegbeli protokoll. A WCF konfigurációja lehetővé teszi, hogy a fejlesztő számára a különbség csak a konfigurációs fájl szintjén jelentkezzen, ezért az implementált szolgáltatások számára mindkét protokollon keresztül nyitottam egy-egy végpontot.\\
A leírt hálózati kapcsolaton közlekedő üzenetek tartalmát az XML-alapú \emph{SOAP} üzenetek (\emph{Simple Object Access Protocol}) határozza meg. A SOAP speciálisan webszolgáltatások számára kifejlesztett, strukturált adatot tartalmazó üzenetekkel kommunikáló protokoll, amely a már említett WSDL-hez hasonlóan valamilyen alkalmazásrétegbeli protokoll fölött működik. Egyes webes szolgáltatások számára a SOAP túlságosan merev, összehasonlítva a REST kommunikációs architektúra kínálta rugalmassággal. SOAP használata esetén ugyanis a \emph{service} és a \emph{client} együttes, párhuzamos fejlesztése a legcélravezetőbb a nagyfokú csatoltság miatt. A Raspberry-n futó képfeldolgozó alkalmazás esetében szerencsére ez a helyzet, és a WCF-ben is ez az alap beállítás, így végül majdnem minden megvalósított szolgáltatás a \code{BasicHttpBinding} kötést használja az alapértelmezettől kis mértékben eltérő beállításokkal. A keretrendszer több, HTTP kapcsolat kialakítására szolgáló \emph{binding} objektumot is biztosít (\code{WSDualHttpBinding}, \code{WSHttpBinding}), de azok interoperabilitása korlátozott, így az alap szolgáltatások mellett döntöttem.\\
A SOAP üzenetek technológia független módon képesek üzenet alapú kommunikációra. Három főbb érszből áll:
\begin{itemize}
\item egy \emph{boríték}, mely a az üzenet struktúrát és a feldolgozás módját definiálja
\item az alkalmazás-specifikus adattípusokra vonatkozó kódolás
\item a távoli eljárások hívására és a válaszok formátumára vonatkozó megegyezés
\end{itemize}
A protokoll üzeneteinek struktúrája a \ref{figure:soap_message} ábrán látható. A WCF keretrendszer az üzeneteket a megfelelő \emph{binding}-nak megfelelően a háttérben előállítja, így alapesetben a fejlesztőnek nem kell ismernie, viszont a program Raspbian-ra való integrálása során a nem megfelelő üzenetformátumból adódtak problémák. A SOAP-üzenetek megvizsgálása után a kapcsolat paramétereinek megfelelő változtatása orvosolta a problémát.

\begin{figure}[h]
\centering
\begin{minipage}{0.73\textwidth}
\begin{mdframed}[backgroundcolor=gray!20]
\begin{scriptsize}
\begin{lstlisting}[language=XML]
<?xml version="1.0"?>
<soap:Envelope xmlns:soap=
  "http://www.w3.org/2003/05/soap-envelope">
  <soap:Header>
  	<!-- Header nodes -->
  </soap:Header>
  <soap:Body>
    	<!-- Object properties -->
  </soap:Body>
</soap:Envelope>
\end{lstlisting}
\end{scriptsize}
\end{mdframed}
\end{minipage}
\hspace{0.5cm}
\begin{minipage}{0.21\textwidth}
\includegraphics[height=4.3cm]{soap_message.pdf}
\end{minipage}
\caption{SOAP üzenetek felépítése}
\label{figure:soap_message}
\end{figure}

\section{Szolgáltatások} \label{section:servicek}
A WCF dokumentációja szolgáltatások létrehozására két eltérő módot biztosít: a \emph{service} deklarálása interfészként vagy osztályként. Mindenkét esetben a \code{[ServiceContract]} attribútumot kell alkalmazni az adott típusra, de az újrafelhasználhatóság szempontjából az interfészek használata a javasolt, így a diplomamunkámban a dokumentáció iránymutatását követtem. \cite{wcf-doc} \\
Összesen négy darab \emph{service contract} interfészt definiáltam, amelyek közül egy időben hármat használ az alkalmazás. A szolgáltatás funkcionalitását adó távolról hívható metódusok a \code{[ServiceContract]} interfész azon tagfüggvényei lesznek, amelyeket az \code{[OperationContract]} attribútummal ellátunk. A szoftver struktúrájának szempontjából a definiált szolgáltatások a \code{Rpi.Service} névtérben, a \code{RpiServices} \emph{assembly}-ben lettek deklarálva.\\
\\
Az implementált szolgáltatások közös vonása, hogy mindegyikük tervezéséhez a \emph{singleton} (magyarul egyke) tervezési mintát vettem alapul. Mivel a beágyazott számítógépen a szolgáltatások egyetlen képfeldolgozó folyamat vezérlésére szolgálnak (erre a \emph{use-case} által definiált egyetlen kamera jelenlétéből következtethetünk), így az egyedülálló művelet bizonyos attribútumainak változtatásához is elég egy-egy távolról elérhető szolgáltatás. A \emph{singleton} tervezési minta pontosan azt írja elő, hogy egy osztályt legfeljebb egyszer lehet példányosítani, így egy programban legfeljebb egyetlen adott típusú objektum létezhet.\cite{design-patterns} Gyakorlati megvalósítása a konstruktor priváttá tételével és egy, az osztály típusával megegyező statikus tagváltozó hozzáadásával érhető el. A statikus adattag referencia az egyetlen példányra, vagy \code{null}. A példányhoz a publikus \code{GetInstance()} tagfüggvény segítségével férhetünk hozzá, amely visszaadja a példányt, amennyiben létezik, ha pedig nem, a privát konstruktorral példányosítja és visszaadja azt, miután a statikus referenciát ráállította.\footnote{A tervezési minta által meghatározott tagváltozók és tagfüggvények a fejezet osztálydiagramjain a \ref{fig:PipelineService_class} ábrán kívül nem kerültek megjelenítésre.}\\
A WCF a szolgáltatás kliens oldali hívásakor alapértelmezett esetben minden hívás esetén új példányt hoz létre az azt implementáló osztályból a hívás teljesülésének idejére. Ez nem felel meg a fent leírt elvnek, miszerint a szolgáltatásokat reprezentáló osztályok csak egyszer kerülnek példányosításra. Ezért a \emph{service} osztályok definíciójában be kell állítani, hogy minden, kliensek által kezdeményezett hívás ugyanahhoz a példányhoz fusson be a \code{[ServiceBehavior]} attribútum \code{InstanceContextMode} tagváltozójának \code{Single} értékűre állításával.

\subsection{\code{PipelineService}} \label{subs:pipeline_service}

\paragraph{Feladata} A távoli gépen futó csővezeték konfigurációjára szolgáló programozási felületet definiálja. 

\paragraph{Működése} Az osztálydiagramja a \ref{fig:PipelineService_class} ábrán látható, a szolgáltatás az \code{IPipelineService} \emph{service contract} implementálásával jön létre. Az interfészre, annak definiált tagfüggvényeire és az osztályra is alkalmaztam a \ref{section:servicek} alfejezet bevezetőjében tárgyalt attribútumokat a kívánt működés eléréshez.

\begin{figure}[h]
\vspace{.5cm}
\includegraphics[scale=1.1]{PipelineService.pdf}
\centering
\vspace{.2cm}
\caption{\code{PipelineService} szolgáltatás}
\vspace{.5cm}
\label{fig:PipelineService_class}
\end{figure}

Látható, hogy az osztály egyik tagváltozója egy \code{OperationPipeline} típusú csővezeték, a kliens által kért változások ezen keresztül jutnak érvényre a képfeldolgozásban.\\
A működés során szükség van új művelet hozzáadására a feldolgozási sorhoz, ez egy \emph{Operation} objektum átküldését jelenti a klienstől a szolgáltatásnak. Szerencsére a \emph{cv4s} keretrendszerben definiált osztály támogatja a sorosítást, így hálózati kapcsolaton keresztül is átküldhető. A szerializáció során a megjelölt tagváltozók XML formátumba íródnak, amely alapján a szolgáltatás példányosítani tudja a kapott paraméterekkel a saját \code{Operation} példányát. Sajnos a \emph{cv4s} keretrendszer nem biztosít beépített eszközt az objektum példányosítására pusztán az XML formátumú adat alapján (\emph{deserialzation}-re), így saját osztályt implementáltam a feladat elvégzésre, ennek rövid bemutatása a \ref{subs:op_factory} alfejezetben olvasható.\\
A kliens számára elérhető funkciók a következők:
\begin{itemize}
\item \code{AddOperation(string xml, bool repeatable)} Hozzáadja a paraméterként kapott string típusú, XML formátumban adott műveletet, a csővezeték \code{OperationList}-jéhez, a szintén paraméterként kapott \emph{bool} paraméter függvényében ismétlődő, vagy egyszer végrehajthatóként.
\item \code{InsertOperation(string xml, bool repeatable, int index = -1)} Az előző metódus funkcionalitását egészíti ki egy index argumentummal, ami beszúrandó művelet helyét jelöli ki a listában. Ha az index negatív értékű, vagy nagyobb, mint a listaelemek száma, akkor a végére szúródik be. Valójában az \code{AddOperation} metódus is ezzel a függvénnyel van implementálva, csak az olvashatóság kedvéért került bele redundáns módon a két tagfüggvény.
\item \code{RemoveOperation(int index=-1)} Eltávolítja a megadott indexű műveletet a konténerből.
\item \code{UpdatePipeline(List<string> operationXmls)} A műveleteket tartalmazó teljes listát a paraméterként kapott műveletekkel tölti fel azok példányosítása után.
\item \code{uint Fps(int fps)} Megpróbálja beállítani a feldolgozási sebességet a kapott paraméter értékére, és visszaadja a valójában beállított értéket a kliensnek.
\item \code{Ping()} a metódus a szolgáltatás kliens oldali elérhetőségének ellenőrzésére szolgál, megfelelő kapcsolat esetén a \emph{service} verziószámát adja vissza.
\end{itemize}

\paragraph{Konfiguráció}

Az elkészült szolgáltatáshoz tartozó végpontok konfigurációját adminisztratív módon definiáltam a Visual Studio projekthez tartozó \code{App.config} fájlban. A WCF által biztosított \code{BasicHttpBinding} összeköttetést használtam a HTTP protokollt használó végponthoz, az időkorlátok (fogadási, küldési és a kapcsolat lezáráshoz tartozó) értékeinek 10 másodpercre való változtatásával. A \emph{service} elérhető még TCP protokollal is a \code{NetTcpBinding} osztályt példányosítva, valamint metaadat cseréhez definiáltam még egy MEX (\emph{Metadata EXchange}) végpontot is \code{MexHttpBinding}-gal. A fejlesztés során a \code{localhost} (127.0.0.1) címet használtam a szolgáltatás elérésre a lokális gépen, amely így a következő címeken érhető el:

\begin{mdframed}[backgroundcolor=gray!20]
\begin{normalsize}
\begin{lstlisting}[language=XML]
http://127.0.0.1:16480/OperationPipeline/
net.tcp://127.0.0.1:16480/OperationPipeline/
\end{lstlisting}
\end{normalsize}
\end{mdframed}

\subsection{\code{OperationFactory}} \label{subs:op_factory}

Mivel a \emph{cv4s} keretrendszer nem támogatja a képfeldolgozó műveletek deszerializációját pusztán XML fájl alapján, szükség volt egy erre alkalmas osztály implementálására. Az \code{OperationFactory} osztály a \emph{factory} tervezési mintát megvalósítva példányosít \code{Operation} objektumokat azok sorosított XML reprezentációja alapján.\cite{design-patterns} Az osztálydiagram az \ref{fig:OperationFactory_class} ábrán látható, új \code{Operation} objektum az osztály \code{GetOperation} metódusának hívásával példányosítható, paraméterként az XML reprezentációt adva.

\begin{figure}[h]
\vspace{.5cm}
\includegraphics[scale=1.1]{OperationFactory_class.pdf}
\centering
\vspace{.2cm}
\caption{\code{OperationFactory} osztálydiagramja}
\vspace{.5cm}
\label{fig:OperationFactory_class}
\end{figure}

A kapott szöveges adat fejlécében megtalálható a létrehozandó objektum típusa, amelyet a \emph{factory} objektum a .NET keretrendszer \code{XmlDocument} osztályának segítségével keres ki. Az osztálynév ismeretében a \code{GetOperation} metódus egy \code{switch-case} szerkezettel kiválasztja, pontosan milyen típusparaméterrel kell meghívni a \emph{cv4s framework} \code{XmlSerializationHelper} objektumának deszerializáló tagfüggvényét és visszaadja az eredményül kapott objektumot. Ha az osztálynevet nem ismeri az \code{OperationFactory}, a képfeldolgozó műveleteket tartalmazó \emph{assembly} megfelelő típusát példányosítja a .NET keretrendszer \code{Activator} osztályával, azonban ilyenkor a kapott paramétereket nem veszi figyelembe, az adott \code{Operation} típus alapértelmezett konstruktorát hívja meg.

\subsubsection{Tesztek}

A kommunikációt végző komponensek közül egyedül ehhez az osztályhoz készült \emph{unit test}, lévén a többi valamilyen webes szolgáltatást valósít meg, így a gyakorlatban komponens tesztet érdemes rájuk készíteni.\\
A teszt menete viszonylag egyszerű, első lépésben példányosít egy \code{Thresholding} műveletet, majd XML formátumba sorosítja. Az \code{OperationFactory} segítségével létrehoz egy új objektumot a soros adatformátum alapján, majd ellenőrzi, hogy a két \code{Thresholding} változó tagváltozói megegyeznek-e.

\subsection{\code{RpiLogService}} \label{subs:rpi_log_service}
\paragraph{Feladata} A végrehajtó eszköz távoli felügyeletét teszi lehető a feldolgozás során keletkező üzenetek továbbításával. A funkciót ellátó osztály megvalósítja a \emph{cv4s} \code{ILog} interfészét, így rugalmasságot biztosít a felhasználónak, ha esetleg nem távoli elérhetőséget lehetővé tévő alkalmazásban szeretné az \code{OperationPipeline} modult használni. Az interfész különböző szintű (\code{verbose}, \code{debug}, \code{information}, \code{warning}, \code{error}) üzenetek továbbítását írja elő a \ref{subs:logging} alfejezetben leírt funkcionalitás megvalósítására.
\paragraph{Működése} A webes szolgáltatások \emph{service-client} architektúrájában a kliens kérésére küld adatot a \emph{service}, viszont jelen esetben fordított irányú kommunikációra van szükség, hiszen a kliens nem tudja mikor születik új naplóbejegyzés a túloldalon, így nem is tud célzottan kérést indítani. A WCF keretrendszer alkalmazásával lehetséges úgynevezett duplex kapcsolat kialakítása, amellyel a \emph{service} is meg tudja hívni a csatlakozott kliens elérhető metódusait, ez a WCF terminológiában a \emph{duplex binding}. Ilyenkor két ellentétes irányú HTTP csatorna jön létre a komponensek között. A keretrendszer \code{WsDualHttpBinding} objektumát használva meglehetősen egyszerűen használható ilyen kapcsolat. A szoftveres megvalósításhoz a \emph{contract}-ot a \ref{section:servicek} alfejezet bevezetőjében leírtaktól eltérően kell implementálni, a szolgáltatásnak ugyanis szüksége van a kliens oldalon hívható metódusok ismeretére egy úgynevezett \emph{callback contract} által. Ez a szerződés szintén egy C\# interfész, amelynek tagfüggvényeit a \code{[ServiceMethod]} attribútummal látjuk el, és a szolgáltatás \code{[OperationContract]} attribútumának paraméterül adjuk. Az implementált osztályok és interfészek viszonyát az \ref{fig:RpiLogService_class} ábrán látható osztálydiagram szemlélteti.\\

\begin{figure}[h]
\vspace{.5cm}
\includegraphics[scale=1.1]{RpiLogService.pdf}
\centering
\vspace{.2cm}
\caption{\code{RpiLogService} osztálydiagramja}
\vspace{.5cm}
\label{fig:RpiLogService_class}
\end{figure}

A kliens oldali \emph{callback contract} (\code{IRpiLogServiceCallback}) az \code{ILog} interfész tagfüggvényeit fedi fel a szolgáltatás számára, a beérkező naplóüzenetek hatására ezeket hívja meg a \emph{service}. A kliens oldalnak a szolgáltatás igénybevételéhez rendelkeznie kell egy olyan objektummal, amely a \emph{callback} függvényeket implementálja.\\
A \emph{service} oldalon a naplózást a \code{RpiLogService} típusú objektum látja el. Ez a már említett \emph{singleton} tervezési mintával készült, és a \emph{service contract} mellett implementálja az \code{ILog} interfészt is, hogy többek között az \emph{OperationPipeline} tagváltozójaként is használható legyen. Az ősosztály biztosítja a lokális konzolra történő naplózást, amely ugyan szintén implementálja az \code{ILog} interfészt, azonban a C\# szabályai szerint a leszármazott osztályban szintén szükséges megadni a viszonyt. \cite{cs-in-a-nutshell}

A naplózási szolgáltatásra a távoli eszköz \code{Subscribe} metódusával iratkozhatnak fel. A feliratkozás során a \code{Channel} objektumot, amin keresztül a feliratkozást kérvényező hívás érkezett, az \code{RpiLogService} elmenti egy konkurens hozzáférést biztosító asszociatív tömbbe, amelyet egyedi 128 byte-os azonosítóval tud megcímezni; ez az azonosító minden új kliens feliratkozásakor a \code{Guid} osztály segítségével generálódik. Azért van szükség a versenyhelyzetek elkerülésére, mert a \emph{service} tetszőleges számú klienst szolgálhat ki adott időpillanatban és elméletileg előfordulhat, hogy két kliens egyszerre akar feliratkozni a szolgáltatásra. Amennyiben egy kliens nem akar több naplóüzenetet fogadni, az \code{Unsubscribe} függvény WCF-en keresztül való hívásával iratkozhat le, ekkor az őt azonosító bejegyzést a \emph{service} törli. Duplex kapcsolat esetén szolgáltatás nem tudja automatikusan detektálni, hogy a kliens leiratkozott \cite{wcf-doc}, ezért ha adott csatornán a konfigurációs fájlban megadott ideig nem kap választ, törli a kliens csatornáját az ismert kapcsolatok közül.\\
A működés során a bejövő naplóbejegyzések hatására először az ősosztály megfelelő naplózó metódusa hívódik, hogy a lokális konzolon is megjelenjen az üzent, majd az \code{InvokeClientCallback()} függvényhívás az ismert kliensek \code{Channel} objektumára továbbítja a naplóbejegyzést, kiváltva a kliens oldalon a megfelelő \emph{callback} függvényt. Az ábrán csak mellékes megjegyzésként szereplő \code{RpiClientLogger} a kliens oldalon naplózást végző osztály, részletes leírása a \ref{subs:client_log} bekezdésben található.

\paragraph{Konfiguráció} A HTTP protokoll felett történő kommunikációra a WCF már említett \code{WsDualHttpBinding} kötést használtam. A \code{PipelineService} modul beállításaival ellentétben itt sajnos nem volt lehetőség TCP kapcsolat kialakítására, mert a WCF nem biztosít ehhez, megfelelő \emph{binding} objektumot az előre definiáltak közül. Természetesen implementálni lehetne megfelelő tulajdonságokkal rendelkező osztályt, ami a duplex kapcsolatot kezelni tudná, azonban egyszerűbb megoldásnak tűnt a következő bekezdésben bemutatott szolgáltatás létrehozása.

\subsection{\code{PollLogService}}
\paragraph{Feladata} Az alkalmazás fejlesztése során a működést nagyrészt Windows operációs rendszeren vizsgáltam a beágyazott eszközön csak viszonylag ritkán futtatva a programot az \ref{subs:fejlesztes_menete} alfejezetben leírtak szerint. Emiatt azonban a platformtól való függőségek csak egy-egy nagyobb részlet elkészülte után derültek ki, így a figyelmemet elkerülte, hogy az \ref{subs:rpi_log_service} alfejezetben bemutatott \code{WsDualHttpBinding} csak Windows környezetben működik, a Raspberry/Raspbian platformon nem. A probléma megoldására olyan naplózó komponenst kellett tervezni, ami a megszokott, egyirányú metódushívással képes az üzeneteket a klienseknek továbbítani, így született meg a \code{PollLogService} osztály, ami \emph{polling} módszerrel periodikusan kéri le az üzeneteket a \emph{service}-től. A két megvalósított naplózó szolgáltatás közül egyszerre csak az egyiket lehet használni, mert az objektumok, amelyek a naplóbejegyzéseket készítik, csak egyetlen példányra tartalmaznak referenciát a definíciójuk alapján. Így még a szolgáltatások közzé tétele előtt kell kiválasztani a naplózó szolgáltatás típusát az \ref{subs:cli} bekezdésben leírtak szerint.

\begin{figure}[h]
\vspace{.5cm}
\includegraphics[width=1.0\textwidth]{PollLogService.pdf}
\centering
\vspace{.2cm}
\caption{\code{PollLogService} osztálydiagramja}
\vspace{.5cm}
\label{fig:PollLogService_class}
\end{figure}

\paragraph{Működése} A naplózó szolgáltatás a \code{RpiLogService}-hez hasonlóan a \emph{publish-subscribe} sémát követi, azaz a kliensek a feliratkozás után értesülnek a bejegyzésekről. Itt azért van szükség a feliratkozásra, mert a \emph{service} a kliensek számára külön tárolókban gyűjti az üzeneteket, és mikor a kliens lekéri, egyszerre elküldi az addig összegyűlt bejegyzéseket, majd üríti a megfelelő tárolót. Ehhez a funkcióhoz szükség volt egy konténer osztály implementálásra, ez a \code{LogQueue}, amely a különböző szintű üzeneteket egy-egy várakozási sorban tárolja, a \emph{composition over inheritance}\footnote{Olyan szoftvertervezési alapelv, amely az osztályok polimorf viselkedésének megvalósításához az örökléssel szemben azok kompozícióját részesíti előnyben\cite{design-patterns}} elvet szem előtt tartva, és konkurens hozzáférés ellen is védelmet nyújt a konzisztens adatállapot fenntartása érdekében. Versenyhelyzet úgy alakulhat ki a szálak között, hogy több szálon futó objektumok írnak a naplózó objektumba, így garantálni kell az inkonzisztens állapot elkerülését a standard .NET típusuk segítségével. A kliens feliratkozásakor az \ref{subs:rpi_log_service} bekezdésben leírtakkal azonos módon új azonosítót kap, és bekerül az ismert kliensek közé. Amikor a kliens lekéri valamelyik naplószinthez tartozó bejegyzéseket, a hozzá tartozó \code{LogQueue} példány megfelelő konténerét kapja meg, ami a \emph{service} oldalon ürítésre kerül.

\paragraph{Konfiguráció} Lévén a szolgáltatás a \emph{request-reply} kommunikációs sémát követi lehetőség volt az \ref{subs:pipeline_service} alfejezetben leírt konfiguráció használatára. A HTTP végpont így a \code{BasicHttpBinding} objektumot használja. Az \code{RpiLogService}-től eltérő módon TCP végpontot is képes nyitni, amelyet a \code{NetTcpBinding} osztály példánya kezel.

\begin{figure}[h]
\centering
\begin{minipage}{1\textwidth}
\begin{mdframed}[backgroundcolor=gray!20]
\begin{normalsize}
\begin{lstlisting}[language=XML]
http://127.0.0.1:16480/PollLog/
net.tcp://127.0.0.1:16480/PollLog/
\end{lstlisting}
\end{normalsize}
\end{mdframed}
\end{minipage}
\caption{\code{PollLogService} szolgáltatás elérése a lokális gépen}
\end{figure}

\subsection{\code{SnapshotService}}
\paragraph{Feladata}
A képfeldolgozó csővezeték működésének ellenőrzésre szolgál a folyamatból kiragadott képek lekérdezésével. A felhasználó így megvizsgálhatja, hogy az adott konfiguráció mennyire felel meg céljainak. 

\paragraph{Működése}
A szolgáltatáshoz tartozó \emph{service contract} mindössze egyetlen metódust tartalmaz, a képek átküldésére alkalmas \code{GetSnapshot()} függvényt. A függvény fejléce a következő:
\begin{mdframed}[backgroundcolor=gray!20]
\begin{small}
\begin{lstlisting}[language=java]
byte[] GetSnapshot(Cv2.ImreadModes mode);
\end{lstlisting}
\end{small}
\end{mdframed}

Képek sorosítására igyekeztem egy segédosztályt létrehozni, amely a WCF \emph{data contract} nevű, osztályok továbbítására vonatkozó egyezményét használja. A \code{MatSerializationHelper} osztály feladata az lett volna, hogy elfedje a sorosítási és visszaalakítási műveleteket, azonban a modul használatával nem sikerült adatot átküldeni az eszközök között. Így végül a képet a kliens \code{byte[]} formátumban kapja és az OpenCvSharp használatával alakítja vissza képi objektummá.

\paragraph{Konfiguráció} 
A szolgáltatást valójában csak a konfigurációja miatt kellett a \\ \code{PipelineService}-től külön osztályban implementálni. Az ok az adat eltérő típusa, ugyanis a konfigurációval kapcsolatos üzenetek mind szöveges formátumban vannak, digitális kép esetében azonban a bináris adatot továbbítunk. További eltérés az adatmennyiség, ami az időkorlátok konfigurációját befolyásolja. A WCF keretrendszer bináris adatok továbbítására támogatja a W3C konzorcium MTOM (\emph{message transmission optimization mechanism}) nevű kódolását, így a képek küldésére ezt alkalmaztam. A már említett időkorlátokat a szöveg alapú üzenetektől eltérően 30 másodpercre állítottam, valójában heurisztikus alapon, mivel valódi alkalmazásban a hálózati kapcsolat sebessége és a kamera felbontása határozza meg az értéket.

\section{Host alkalmazás} \label{section:RpiRemoteHost}

A távoli eszközön futó program egyszerű \code{Console Application}, amelynek feladata a parancssorból kapott paraméterek \ref{subs:cli} bekezdésnek megfelelő feldolgozása, és egy \code{RpiRemoteHost} osztály példányosítása ezen paraméterekkel. Az \code{RpiRemoteHost} a \ref{chapter:kepfeldolgozo}. és \ref{chapter:kommunikacio}. fejezet korábbi részeiben leírt funkcionalitások egyetlen programba való összefogását végzi. Ez az osztály felelős az erőforrások inicializálásért, a csővezeték és a kamera olvasó osztály felügyeletéért, valamint a végpontok kezeléséért.\\
\\
A megvalósított osztály a három egyszerre elérhetővé tett végponthoz tartozó \code{ServiceHost} típusú tagváltozót tartalmaz, a típust definiáló osztály a WCF keretrendszer része, feladata, hogy a szolgáltatásokat elérhetővé tegye a konfigurált címeken. A konstruktora több lehetőséget is ad a példányosításra, az \code{RpiRemoteHost} osztály minden végpont esetében a \emph{service} példányát adja át a \emph{host}-nak. Erre azért van szükség, mert a szolgáltatásokat alapesetben a \emph{host} objektum példányosítja, azonban azokat a megnyitásuk előtt konfigurálni kell.

\begin{itemize}
\item A naplózó szolgáltatások példányosítása azért szükséges, mert ezt az objektumot használja az összes \emph{service} oldali komponens, így a rá mutató referenciára mindenképp szükség van.
\item \code{OperationService}: a \code{OperationPipeline}, \code{CameraReader} és \code{Log} tagváltozók inicializálása a kapott paraméterek alapján.
\item \code{SnapshotService}: naplózó tagváltozó beállítása.
\end{itemize}

Az \code{RpiRemoteHost} inicializálásakor megkapja a parancssorban megadott paramétereket és azok alapján hozza létre szolgáltatásokat. Az \code{Open()} metódus publikálja a létrehozott műveleteket \emph{service}-eket, a \code{Close()} pedig bezárja azokat.

\subsection{Parancssori interfész} \label{subs:cli}

Mivel az eszköz a felhasználói eset szerint nehezen hozzáférhető helyen található, gondoskodni kellett a távirányítható alkalmazás távolról, konfigurálható paraméterekkel történő indításáról is. Előfordulhat ugyanis, hogy a program olyan állapotba kerül, amelyből már csak az újraindításával lehet visszatérni az elvárt működéshez. Ehhez készítettem egy, a GNU/Linux operációs rendszeren megszokott paraméterstruktúrával rendelkező \emph{CLI}-t (\emph{command line interface}-t). A leírt paraméterekkel összeállított parancsot távoli gépről \code{ssh} paranccsal adhatjuk ki az eszközön. A szabványos .NET Framework-ben sajnos nincs hasonló parancssori támogatást nyújtó osztálykönyvtár, így a \code{Mono.Options}  nevű csomagot használtam -- amely a Mono keretrendszer része, és ennek megfelelően nyílt forráskódú --, ennek segítségével egyszerűen elkészíthető készíthető következő paraméterekkel rendelkező interfész:

\begin{figure}[h]
\centering
\begin{minipage}{1\textwidth}
\begin{mdframed}[backgroundcolor=gray!20]
\begin{scriptsize}
\begin{lstlisting}[language=xml]
  -l, --location=VALUE       Path to the video file to play.
  -p, --poll                 Use the poll log service. If false, duplex binding is
  			     required.
  -v, --verbosity=VALUE      Use this verbosity level.
  -f, --fps=VALUE            Framerate for the camera reader.
  -w, --winPort=VALUE		 Use this value to read from camera port on Windows
  -h, --help                 Show this message and exit
\end{lstlisting}
\end{scriptsize}
\end{mdframed}
\end{minipage}
\caption{CLI segítség menüje} \label{figure:cli_help}
\end{figure}

Az "-l" kapcsolóval a fájl helyét lehet megadni, ahol a kiolvasandó videostream található, ez lehet Linuxon eszköz helye, vagy egy létező videofájl a fájlrendszerben.
A "-w" argumentummal megadott szám annak az csatlakoztatott kamera eszköznek a számát jeli ki, amelyből a kamera adatokat olvassuk ki Windows rendszeren, ezt az opciót akkor adjuk meg, ha a fájl helyét nem jelöltük meg a -l kapcsolóval.
A "-p" paraméterrel azt szabályozhatjuk, hogy a \emph{service}-ünk naplóüzenet szolgáltatása kétirányú kapcsolatot nyisson vagy a kliens által lekérdezendő legyen.
A "-v" kapcsoló mellé szükséges egy pozitív egész szám, ami a \emph{service} oldali naplózás szintjét állítja be. Alapesetben érdemes az alapértelmezett "\emph{verbose}" szinten hagyni, és a kliens oldalon kiszűrni a szükségtelen üzeneteket. 
Beállítható még továbbá a csővezeték indításakor használt kiolvasási sebesség, ami képkocka/másodperc dimenzióval értendő.
A program paramétereiről összefoglaló információt a "-h" kapcsolóval kérhetünk, ekkor a \ref{figure:cli_help} ábrán látható üzenetet kapjuk.

\section{Fejlesztés menete} \label{subs:fejlesztes_menete}

A fejlesztés során nehézséget jelentett, hogy a kódot Visual Studio fejlesztőkörnyezetben írtam a kódot, azonban egy másik, eltérő operációs rendszerű és architektúrájú eszközön kellett tesztelni. Emiatt a fejlesztés során a programot a lokális, Windows rendszeren futtattam, az eszközön viszonylag ritkán próbáltam ki. A szolgáltatások mindegyike a \code{16480} számú porton elérhető, azonban a Windows nem engedélyezi a nem adminisztrátori jogokkal futtatott programok számára -- néhány szabványos kivételével -- a portok lefoglalását. Így az alkalmazást vagy adminisztrátorként kell futtatni (ami meglehetősen körülményes módszer), vagy a hálózati interfész használatát engedélyezzük a felhasználói programok számára. Ehhez a Windows parancssorában adminisztrátorként a következő parancsot kell kiadni, a \code{user} paraméternek pedig azt felhasználót megadni, akinek a port használatát engedélyezzük.

\begin{mdframed}[backgroundcolor=gray!20]
\begin{small}
\begin{lstlisting}[language=XML]
netsh http add urlacl url="http://+:16480/" user=...
\end{lstlisting}
\end{small}
\end{mdframed}

A Raspberry Pi használatához a \emph{cygwin} nevű, Linux terminált emuláló programot használtam, ezen keresztül \code{ssh}-val jelentkeztem be az eszközre, így a kiadott parancsok a távoli számítógépen hajtódtak végre. Az eszközön való tesztelést nehezítette, hogy a Visual Studio-val lefordított binárisokat minden futtatás előtt az eszközre kellett másolni. A másolást szintén a \emph{cygwin} terminálból, az scp paranccsal végeztem, majd az eszközre való bejelentkezés után futtattam a programot. A képfeldolgozó műveletek eredményességéhez szükséges a grafikus felhasználói felület is, ehhez a \emph{VNC Viewer} nevű alkalmazást használtam, ami szintén \code{ssh} kapcsolaton keresztül jeleníti meg az eszköz GUI-ját. Ez a folyamat azonban  kellően időigényes ahhoz, hogy a fejlesztés során a teszteléshez más módszer után nézzek. \\
A program manuális teszteléséhez szükséges idő csökkentésére először létrehoztam egy virtuális asztali Linux környezetet, lefordítva és telepítve a \ref{chapter:architektura} fejezetben leírt csomagokat. A virtuális gép nyilvánvalóan eltér a beágyazott rendszertől, de a debian disztribúciót használva a különbség nem túl nagy ahhoz, hogy fejlesztés közben használható legyen, időnként természetesen a Raspberry Pi-n is ellenőrizve a program működését. Ebben az esetben a binárisok másolgatása sem szükséges a futtatási környezetek között, ha a mappát, amelybe a fordításkor a fájlok kerülnek, megosztjuk a virtuális gép számára.\\
\\
Sajnos a virtuális gépet futtatni meglehetősen erőforrás-igényes és az általa biztosított szolgáltatások jelentős részére nem is volt szükségem. A \ref{subs:docker_bevezeto} bevezető bekezdésben már említett \emph{Docker} pontosan erre a problémára kínál megoldást, így létrehoztam egy, a \emph{service} futtatásához szükséges szoftvereket tartalmazó \emph{Docker} \code{image}-et, a \ref{subs:docker} pontban kifejtett módon. A \emph{container} használatával már viszonylag egyszerűen lehetett a szolgáltatást GNU/Linux rendszeren kipróbálni.

\subsection{Docker} \label{subs:docker}

A Docker szolgáltatás \emph{lightweight}, azaz kis erőforrás igényű, szeparált és hordozható futtatási környezetet biztosít az alkalmazások számára. A host rendszertől elkülönített környezetet a Docker terminológiája \emph{container}-nek nevezi, ebbe "csomagolva" az alkalmazás függetlenül futhat. Működése során az eredetileg GNU/Linux rendszerre készült program az operációs rendszer csomagjait is használja, nem virtualizálja a hardvert. Windows alatt is futtathatunk Linux konténereket, ezekhez a program egy kis erőforrás igényű virtuális \emph{Alpine} Linux rendszert indít el. 
\\
 A \emph{container}-ek úgynevezett \emph{image} fájlokból indíthatóak a program parancssori interfészével, ezek hasonlóak a virtuális gépeknél használt image-ekhez, a futási környezet tulajdonságait tartalmazzák. Egy \emph{image}-ből tetszőleges számú független \emph{container} indítható különböző paraméterekkel. Image fájlokat a Docker által üzemeltetett \code{hub.docker.com} címen elérhető szolgáltatásról tölthetünk le, vagy elkészíthetjük a személyre szabott image-ünket egy \emph{Dockerfile} alapján. A futtatókörnyezet viszonylag különleges csomagigényei miatt a hordózhatóságot szem előtt tartva az utóbbi megoldás mellett döntöttem.\\
A \emph{Dockerfile} egy szöveges fájl, mely soronként tartalmazza a környezet létrehozásához szükséges utasításokat. A parancssorban a \code{docker build} parancs kiadásával készíthetünk image-et a Dockerfile-ból. \cite{docker-doc}

\begin{mdframed}[backgroundcolor=gray!20]
\begin{small}
\begin{lstlisting}[language=XML]
docker build -t cv4s_rpi -f .\rpi-docker-image.Dockerfile .
\end{lstlisting}
\end{small}
\end{mdframed}

A \code{-t} kapcsolóval a generált image címkéjét, a \code{-f}-fel a Dockerfile helyét adjuk meg. Lehetőség van egy mappa elérési útjának megadására is, amelyben található fájlokhoz a fordítás során hozzáfér a program. Az image készítése közben a Dockerfile minden sora új átmeneti képfájlt készít, és a következő lépés ezt alapul véve, mintegy új réteget hozzáadva készíti el saját átmeneti image-ét. Ez a megoldás a hosszú fordítási idejű képfájlok esetén azért előnyös, mert a folyamatbeli hiba esetén nem szükséges az elejéről kezdeni a \emph{build} folyamatot, az a legutolsó elkészült átmeneti példánytól tud folytatódni. Hátrányos lehet a sok lépést tartalmazó Dockerfile-ok fordítása során, hogy sok átmeneti képfájl keletkezik, ami nagy lemezterületet foglal a háttértárolón, Dockerfile-ok írása során érdemes a kiadott parancsokat \emph{shell script} formájában futtatni. \cite{docker-doc}\\
Az általam elkészített Dockerfájl is csoportosítva hajtja végre szükséges lépéseket a fordítás során, a teljes Dockerfile  a \ref{fig:docker_commands} ábrán látható.

\begin{figure}[h]
\centering
\begin{minipage}{1\textwidth}
\begin{mdframed}[backgroundcolor=gray!20]
\begin{small}
\begin{lstlisting}[language=XML]
# Latest debian release is the base image
FROM debian

# Root permission is needed inside the container
USER root
# Expose this port to the world
EXPOSE 16480
# Copy shell scripts which will install necessary components
COPY install_dependencies.sh install_opencv.sh \
install_opencvsharp.sh install_mono.sh misc.sh \
update_host.sh /shared/

# Run shell scripts
RUN /bin/bash -c "/shared/install_dependencies.sh"
RUN /bin/bash -c "/shared/install_opencv.sh"
RUN /bin/bash -c "/shared/install_opencvsharp.sh"
RUN /bin/bash -c "/shared/install_mono.sh"
RUN /bin/bash -c "/shared/misc.sh"

# Run bash
CMD "/bin/bash"
\end{lstlisting}
\end{small}
\end{mdframed}
\end{minipage}
\caption{\code{Dockerfile}} \label{fig:docker_commands}
\end{figure}

A fordítás során a legfrissebb \emph{Debian} image-ből kiindulva telepítjük a szükséges csomagokat. A létrejövő \emph{image}-en belül \code{root} jogosultsággal fogunk rendelkezni, ez a telepítést is leegyszerűsíti, mert nem kell kiadni a \code{sudo} parancsot a csomagkezelő műveletek előtt. Ezután az 16480-as portot elérhetővé teszi a host gép számára a \code{EXPOSE} utasítással, majd a \code{COPY} paranccsal a fordító a munkakönyvtárban található telepítő scripteket bemásolja a képfájlba. A fájlok a függelék \ref{rpi-build-script} pontjában található telepítő utasításokat tartalmazzák logikai egységekre felbontva. Ezen kívül tartalmaz még egy \code{update\_host.sh} nevű scriptfájlt is, amely arra szolgál, hogy a tesztelés során a hostgépen lefordított binárisokat a kívánt munkamappába másolja, hogy ne az eredeti helyen módosítsuk a fájlokat.\\
\\
A Docker konténer a következő paranccsal készíthető el az \emph{image}-ből:

\begin{mdframed}[backgroundcolor=gray!20]
\begin{small}
\begin{lstlisting}[language=XML]
docker run -ti --name rpi_host -v path/to/binaries\:/mnt/
 -p 16480:16480 cv4s_rpi
\end{lstlisting}
\end{small}
\end{mdframed}

A "\code{-ti}" kapcsoló kombinációval a konténer standard IO fájljait a egy pszeudo terminálra irányítja át, így a bemeneti parancsokat egy Linux konzolról tudjuk átadni és az utasítás kimeneti is itt fognak megjelenni, mintha egy valódi Linux eszközzel kommunikálnánk. \\
A \code{--name} argumentum paramétere lesz a konténer neve, amivel a használata során hivatkozhatunk rá.\\
Lehetőség van a host számítógép fájlrendszerét megosztani a konténerrel, így az ott tárolt fájlok elérhetők a leválasztott környezeteben. A webes szolgáltatás futtatásához szükség van a legfrissebb lefordított binárisokra, melyek elérési útja a \code{path/to/binaries/}. A konténeren belül ezeket a \code{/mnt/} könyvtárban találjuk.\\
A fenti beállításokon kívül szükség van még a \code{-p} kapcsolóval a konténer és a host portjainak összerendelésére, hogy a gazda számítógép hozzáférjen a \emph{service}-hez. A \code{run} parancs utolsó paramétere a Docker \emph{image} neve, amelyből a \emph{container} készül.\\
\\
A tárolón belül a \emph{service} alkalmazás a \code{mono}-val futtatható, a \ref{subs:cli} bekezdés szerinti paraméterezéssel.