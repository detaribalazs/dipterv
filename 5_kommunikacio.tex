\chapter{Kommunikáció}
\section{WPF választása}
Az eszköz távirányíthatóságát a .NET Framework \emph{Windows Communication Foundation} (WCF) nevű, szolgáltatás-orientált keretrendszerének segítségével oldottam meg. A WCF architektúrában a Raspberry Pi-n futó képfeldolgozó alkalmazás a \textit{service} szerepét tölti be, a felügyelő feladatot ellátó, asztali gépen futó alkalmazás pedig kliensként veszi igénybe annak szolgáltatásait.\\
A rugalmasságot szem előtt tartva igyekeztem teljesen elkülöníteni a távoli vezérelhetőséget, illetve a képfeldolgozási funkciókat biztosító osztályokat.\\
Az alább bemutatott szolgáltatások a egy-egy \code{ServiceContract} attribútummal ellátott interfész implementálásával, illetve konfigurációs fájl hozzáadásával válnak távolról elérhető szolgáltatásokká.

\section{Megvalósított \emph{service}-ek és konfigurációjuk}
\subsection{PipelineService}

\paragraph{Szolgáltatások}
A képfeldolgozó modul távoli vezérelhetőségét a \code{PipelineService} osztály biztosítja azáltal, hogy megvalósítja a a \code{IPipelineService} \textit{szerződését}. Ezáltal a következő műveletek lesznek elérhetőek interneten keresztül:
\begin{itemize}
\item művelet hozzáfűzése a végrehajtási lánc végéhez
\item művelet beszúrása megadott pozícióba a műveleti listába
\item művelet eltávolítása a műveletek közül
\item műveleti sor teljes frissítése egy lista alapján
\item képernyőkép lekérése az utolsó feldolgozott képkockával
\end{itemize}

A kliens oldalon konfigurált képfeldolgozási művelet \code{xml} formátummá konvertálódik a küldés előtt, erre a \emph{cv4s} \code{XmlSerializationHelper} modulja biztosít lehetőséget. A szerver oldalon szükség volt visszaalakítást végző, \code{OperationFactory} nevű osztály implementálására. Az osztály az \code{xml} fájl fejléce alapján határozza meg, milyen típusú művelet érkezett, és a \code{Factory} tervezési minta alapjánállítja elő az objektumot.\\
Képek küldése esetén a sorosítást az OpenCvSharp végzi el, az adat \code{byte} tömbként kerül továbbításra.

\paragraph{Konfiguráció} A szolgáltatás konfigurációját a .NET Framework által biztosított, \code{Web.config} fájl módosításával szabtam a szolgáltatás igényeire. A konfigurációs fájl egy \code{xml} fájl, amiben a \emph{service} paraméterei állíthatók be. Ez a fájl írja le többek közt, hogy mely szolgáltatásokat biztosítja a szerver, milyen elérési úton, és milyen protokollal. A feladat végrehajtásához \code{basicHttpBinding} kötést választottam, ami az elvárt működést megfelelően kielégítette. Fontos megemlíteni, hogy mivel hálózati kapcsolaton keresztül szükséges viszonylag nagyméretű képfájlok átvitele, a kapcsolat \emph{timeout} és fájlméretre vonatkozó beállításait ennek megfelelően kellett elvégezni.

\subsection{LogService}

\paragraph{Szolgáltatások} A végrehajtó eszköz távoli felügyeletét teszi lehető a feldolgozás során keletkező üzenetek továbbításával. A funkciót ellátó osztály megvalósítja a \emph{cv4s} \code{ILog} interfészét, így rugalmasságot biztosít a felhasználónak , ha esetleg nem távoli elérhetőséget lehetővé tevő alkalmazásban szeretné az \code{OperationPipeline} modult használni. Az interfész különböző szintű (\code{information}, \code{warning}, \code{error}) üzenetek továbbítását írja elő.

\paragraph{Konfiguráció} A \code{PipelineService} modul beállításaival ellentétben itt arra volt szükség, hogy a felügyelt eszközünk küldjön üzenetek, amik a kliens oldalon kerülnek megjelenítésre. A WCF lehetőséget biztosít erre ún. \code{Callback Service}-ek által, amelyek a kliens oldalon eseményeket váltanak ki.

\section{Fejlesztés menete}

A kliens oldali modul elvárt működésének érdekében \code{Service Reference} hozzáadása szükséges a projektünkhöz. Itt a szolgáltatás elérési útjának megadása után megjelenik a kívánt \code{service}, és a VisualStudio automatikusan legenerálja a szükséges kódrészleteket a használatához. Ha nem Visual Studio-t használunk lehetőség van az \code{svcutil} parancssori program alkalmazására, ami elvégzi a szolgáltatásunk kliens oldali kódjának létrehozását. \\
A tesztelés érdekében egy egyszerű parancssori

\section{Docker}

\section{Tapasztalatok a fejlesztés során}
A fejlesztés során nehézséget jelentett, hogy a kódot Visual Studio fejlesztőkörnyezetben írtam a kódot, azonban egy másik, eltérő operációs rendszerű és architektúrájú eszközön kellett tesztelni. Emiatt a létrehoztam egy virtuális asztali Linux környezetet, lefordítva és telepítve a \ref{integracio} fejezetben leírt csomagokat. A kisebb változtatásokat és javításokat Windows és Visual Studio segítségével végeztem, majd bizonyos időközönként a virtuális Linux alatt is leteszteltem a működést. Legritkábban magán a Raspberry Pi-n futtattam a programot, mivel a \code{assembly}-k méretének és számának növekedésével egyre körülményesebbé vált a program \code{ssh}-n keresztüli átmásolása. A másolás után az eszközön a \code{mono} parancs kiadásával indítható el a kiválasztott \code{.exe} fájl. A Raspberry Pi távoli elérését a \textit{VNC Viewer} nevű alkalmazás tette lehetővé.\\
\\
A távoli elérés konfigurációja kezdetben nehezen ment részben a WCF-fel való tapasztalatlanság, részben pedig a szűk hibakeresési lehetőségek miatt. Ezért a konfigurációval kapcsolatos hibák kiküszöbölésére létrehoztam egy \code{DummyService} nevű, "állatorvosi ló" szolgáltatást, amely segítségével az átküldött adatok sorosítását és a \textit{service}-ek konfigurációját lehetett kipróbálni anélkül, hogy a lényeges funkcióval bíró szolgáltatásokat módosítani kellene.\\
\\
Kép objektumok sorosítására igyekeztem egy segédosztály létrehozására, amely elfedi a sorosítási és visszaalakítási műveleteket, azonban valamiért a modul használatával nem sikerült adatot átküldeni az eszközök között.