\chapter{A féléves munka értékelése}

A szemeszter során a feladatkiírásban szereplő pontok körülbelül felét sikerült megvalósítani:
\begin{itemize}
\item portolhatósági lehetőségek körüljárása beágyazott platformokra
\item távirányítható szoftverkomponens implementálása
\item tervezési minták szem előtt tartása a fejlesztés során
\end{itemize}

A féléves munka talán legnagyobb hiányossága, hogy az elkészült szoftvermodulok nem rendelkeznek semmilyen automatizált teszttel, ami a munka előrehaladásával egyre több gondot okozott, főleg kód refaktorálása után. Ilyenkor az egyes funkciókat "kézzel" kellett ellenőrizni, ami jelentős időkiesés, emiatt a következő félévben mindenképpen szükséges lesz a tesztek elkészítése. \\
\\
A felhasználói felületre vonatkozó pontot csak részben sikerült teljesíteni, mivel a szoftver jelenlegi állapotában egy felhasználóbarátnak semmiképp sem mondható konzol felület ad lehetőséget a felhasználói interakcióra, valójában inkább csak a tesztelést teszi lehetővé. \\
\\
A szemeszter során nem jutott idő továbbá az alkalmazásterületre szabott képfeldolgozási funkció implementálására, így ez a részfeladat is a Diplomaterv II. tárgy keretében kerül implementálásra.