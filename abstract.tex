\chapter*{Abstract}

The digital image processing is one of the most popular technologies of the 21st century, it can be extensively applied for -- among others -- the quality assurance of production lines, traffic monitoring and security surveillance of public transport system. After laying down the theoretical foundations in the 1960s, the technology started blooming in the early 2000s by the appearance of hardwares with sufficiently high performance, and after the appliance of the machine learning algorithms in the field, it started to shape increasingly the every day life of people. \\
\\
In the short history of informatics, the pattern of always competing trends can be observed in the developement. As a result, the inner organizing power of the discipline leaves only a few of rival technologies behind, as it happend with the computer hardware architectures, operating systems, and even with the different approaches in the software developement. It is a common challenge in the engineering practice to synchronize the different, with each other incompatible systems or components. In the following thesis, I intended to integrate a software implemented in \emph{.NET Framework} with an embedded computer, which runs Linux operating system, into one application.\\
\\
The dissertation contains the design and the considerations made during the implementation of an image processing software system, which can run on desktop and embedded environment and can be supervised over network connection. Furthermore, the related technologies will be presented, like the \emph{Windows Presentation Foundation} and \emph{Windows Communication Foundation}. \\
\\
The implemented system is a suitable solution for general tasks, that require real-time image processing run on embedded system; it can be used as a solution for specific challenges by utilizing its wide range of configuration parameters.
\\

Besides that, the described solutions and considerations may serve as a starting point for projects in the future, which aim the disengagement of the \emph{cv4s} framework from the desktop environment. 