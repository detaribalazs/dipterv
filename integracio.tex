\chapter{A keretrendszer integrációja}

\section{A \emph{cv4s} függőségei}

Az integráció megkezdése előtt feltétlen szükséges első lépés a \emph{cv4s} keretrendszer függőségeinek felderítése volt, a lehetséges platformok ezek ismeretében határozhatók meg. \\
\\
Lévén a keretrendszer C\# nyelven íródott, így feloldhatatlan függőséggel rendelkezik a \textit{.NET Framework} felé, amely a kód futását biztosító menedzselt környezetet biztosítja. Ez a függőség kis mértékben korlátozza a platformfüggetlenséget, hiszen a keretrendszer 2015 óta nyílt forráskódú, így létezik implementációja a többek között \emph{GNU/Linux} operációs rendszerekre is. Valamelyeset azonban mégis szűkíti a felhasználhatóságot, hiszen feltételezi operációs rendszer jelenlétét az eszközünkön, ami beágyazott környezetben nem evidencia (kisebb komplexitású feladatok egyszerű ütemezővel is megoldhatók). Az így kizárásra kerülő a kisebb teljesítményű mikroszámítógépek azonban valószínűleg egyébként sem lettek volna alkalmasak a valós idejű képfeldolgozásra. \\
A másik \emph{.NET Framework}-függőségből származó korlátozás a \textit{Windows Presentation Foundation} (WPF) osztálykönyvtárat használó komponensek kiesése az integrációból, ugyanis a grafikus megjelenítésért felelős szoftverkomponensek kizárólag \textit{Windows} operációs rendszer alatt érhetők el. A \emph{cv4s} keretrendszer alapvető funkcióit biztosító \code{Core} és \code{Operation} modulok csak kis mértékben függenek a \emph{WPF}-től így azok kis mértékű változtatásával ezek is megszüntethetők.\\
\\
Az OpenCV és az azt közvetlenül használó OpenCvSharp felé jelentkező függőséget szintén fontos megvizsgálni, hiszen ezek feloldása nélkül a keretrendszer integrációja egyáltalán nem lehetséges. Az OpenCvSharp nyílt forráskódú, és bár hivatalosan az x86-os architektúrán kívül nem támogat más processzor típust, azonban keresztfordító alkalmazásával vagy egyből az eszközön fordítva használható ARM alapú processzoron is. A burkoló modul által használt OpenCV binárisok szintén lefordíthatók a ARM architektúrára, hiszen a hivatalos fejlesztői \textit{repostry}-ban megtalálható a szükséges \code{Makefile}, így elvben nincs akadálya, hogy ARM alapú processzort használjunk. \\

\section{Lehetséges platformok}

A fenti megfontolások alapján két lehetséges eszköz merült fel, mint lehetséges célhardver az integrációhoz:
\begin{itemize}
\item RaspberryPi 3 ARM architektúra
\begin{itemize}
\item Windows 10 IoT Core(\emph{Internet of Things}) operációs rendszerrel
\item Raspbian operációs rendszerrel
\end{itemize}
\item Intel Joule x86 architektúra
\begin{itemize}
\item Windows 10 IoT Core operációs rendszerrel
\item Ubuntu operációs rendszerrel 
\end{itemize}
\end{itemize}

Az eredeti, asztali környezettel a legtöbb hasonlóságot az Intel Joule-on futó Windows operációs rendszer mutatja, azonban jelentős hátrány, hogy az Intel már nem gyártja és nem támogatja a platformot.\\
\\
A Raspberry Pi ezzel szemben az egyik legnépszerűbb hardver a kezdő beágyazott fejlesztők között, ennek megfelelően részletes dokumentációval valamint széles közösségi támogatással rendelkezik. Ezen felül a Win10 IoT Core támogatással és saját Linux disztribúcióval is rendelkezik Raspbian néven.

\subsection{Raspberry Pi és Windows 10 IoT Core} \label{rpi-win10iot}

A \emph{.NET Framework} alapú képfeldolgozó keretrendszerünk számára remek környezetet lenne képes biztosítani a Microsoft beágyazott eszközökhöz fejlesztett operációs rendszere, amelyet a miénkhez hasonló IoT alkalmazásokhoz ajánlanak. A hivatalosan támogatott hardverek listáján megtaláljuk a Raspberry Pi 3-at is, és a Raspberry közösségi fóruma is sok segítséget biztosít a fejlesztéshez. \\
Kisebb akadályt jelent, hogy az OpenCV nem biztosít hivatalos \code{Makefile}-t a szoftverkönyvtár lefordításához Win10 IoT Core-ra, ám ez is megoldható. \cite{win10-compile} Ezután az OpenCvSharp szoftvercsomag gond nélkül tudná használni a fordítás eredményeként kapott binárisokat. \\
Megjegyzendő, hogy, az operációs rendszer csak az \emph{Universal Windows Platform} (UWP) alkalmazásokat támogatja, ez a platform váltja le a korábban grafikus felületek létrehozására használt szoftverkönyvtárat, a \emph{Windows Presentation Foundation}-t (WPF), amelyet a \emph{cv4s} is használ. Ennek megfelelően a keretrendszer grafikus felülettel rendelkező alkalmazásai nem használhatók a beágyazott rendszerben, bár ez nem is használati eset.\\
Egy másik, a fejlesztést nehezítő tényező, hogy a Microsoft, bár kínál távoli képernyőelérést biztosító alkalmazást, a kipróbálásának időpontjában csak részlegesen működött, tényleges képernyőképet nem küldött a fejlesztői gépre.

\subsection{Raspberry Pi és Raspbian}

A Raspbian egy Debian alapú operációs rendszer, amit kifejezetten a Raspberry Pi hardverére optimalizáltak. Mivel Linux alatt szükségünk van a futtató környezetre, az alkalmazásunkat Mono segítségével tudjuk használni. A Mono nem implementálja a \ref{rpi-win10iot} alfejezetben említett WPF-et, de ez nem jelent problémát az ott leírt okok miatt. Ezen felül a Mono használata csak kisebb megkötést jelent, amelyekről a későbbi fejezetekben lesz szó.\\
\\
Végső soron a hardvernek és az operációs rendszernek ezt a kombinációját választottam, mint célplatform, egyrészt az OpenCV által biztosított, viszonylag egyszerű fordítási mód miatt, másrészt a korábbi tapasztalatom végett a Linux rendszerekkel. Ezen felül érdekesebb kihívást jelentett egy \textit{.NET Framework}-ben készült alkalmazás integrációja Linux rendszerre, ami nagyobb teret is biztosít a keretrendszer későbbi alkalmazásának akár asztali környezetben futó Linux környezetben is. Továbbá a fejlesztést is megkönnyítette a Raspbian eszköztára (pl. távoli elérés).

\subsection{Intel Joule}

