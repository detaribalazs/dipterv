\chapter{Felügyeleti kliens alkalmazás} \label{chapter:kliens}

A beágyazott képfeldolgozó és a kommunikációt lehetővé tévő szoftverkomponensek megvalósítása során a működés teszteléséhez egy leegyszerűsített, az eszköz távoli vezérléséhez szükséges funkciókat korlátozottan megvalósító konzol alkalmazást használtam. Az \code{RpiTestClient} nevű program billentyűparancsok hatására kommunikált a szolgáltatással, azonban a valódi felhasználók számára gyakorlatilag használhatatlan felület csak a működés tesztelésére volt alkalmas.\\
A kliensek által használt program egy grafikus felület, ami az eszköz távoli vezérlésre és a képfeldolgozási konfiguráció beállítására kínál kényelmes, egyszerűen érthető interfészt.

\section{Tulajdonságok}

A \emph{cv4s} keretrendszer a szolgáltatásait igénybe vevő programoknak grafikus felület készítésére is támogatást nyújt. A keretrendszer a bevezető \ref{ssection:wcf_intro} pontjában már említett, WPF (\emph{Windows Communication Foundation}) \emph{framework}-jét használja, ez meghatározza a kliensprogramok futtatásához szükséges platformot, ugyanis a WPF szolgáltatásai csak Windows operációs rendszeren érhetőek el. \\
A felhasználó felülettel kapcsolatos követelmény volt, hogy a képfeldolgozást végző eszköz konfigurációját módosítani lehessen rajta keresztül. Ezen kívül egy másik funkcionalitást is implementáltam, a kliens alkalmas konfiguráció lokális összeállítására. A helyi gépen történő konfigurációhoz szükséges egy, a folyamatról készült videó fájl, amit a program lejátszik és a beállított képfeldolgozási műveleteket végrehajtja a képkockákon. Miután a konfiguráció a lokális környezetben kielégítő eredményt mutat, a Raspberry Pi-re egy kattintással feltölthető.\\

\begin{figure}[h]
\vspace{.5cm}
\includegraphics[width=1\textwidth]{client-local.png}
\centering
\vspace{.2cm}
\caption{Lokális konfiguráció}
\vspace{.5cm}
\label{fig:client-local}
\end{figure}

A lokális konfigurációs ablak a \ref{fig:client-local} ábrán látható. A felső sávban található "\emph{Path}" vezérlő blokkban lehet kiválasztani a betölteni kívánt videófájlt. A mappa ikonra kattintva megjelenik egy dialógusablak, amellyel a fájlrendszerben böngészhetünk. A videó kiválasztása után a vezérlősáv bal oldalán lévő "\emph{Play}" gombbal indíthatjuk el a lejátszást, ez az indítás után "\emph{Pause}"-ra változik a szüneteltetéshez. A videó a "\emph{Stop}" gombbal állítható le, lejátszásának sebességét pedig az "FPS" mezőben állíthatjuk be.\\
A műveletek beállításához a "\emph{Configuration}" csoportban lévő elemekkel vezérelhetjük. Új képfeldolgozó művelet az "\emph{Add Operation}" gombra kattintva adhatunk a feldolgozási sorhoz, ekkor a \ref{fig:operation-selector} ábrán látható ablak jelenik meg. Itt lehet meghatározni a sorrendet és eltávolítani a szükségtelen műveleteket. A főablakban egy legördülő listában leeht kiválasztani a műveletet konfigurálásra, ekkor az bal alsó oldalán megjelennek a konfigurálható tulajdonságok. Ezt az Xceed nevű vállalat \code{DataGrid} objektuma teszi lehetővé. A konfigurációs ablak fölötti mezőben a feldolgozáshoz tartozó naplóüzenetek jelennek meg egymás alatt időrendben.\\
Az Raspberry Pi távoli felügyeletéhez tartozó beállítások a menüsorban a "\emph{Remote}" fül alatt érhetőek el, a képernyőkép a \ref{fig:client-remote} ábrán látható. A webszolgáltatás IP címe és a portszámának megadása után a "\emph{Connect}" gombra kattintva csatlakozunk hozzá (csatlakozás után az ábrán látható "\emph{Disconnect}" feliratra cserélődik), ezután érhetőek a távoli felügyeleti lehetőségek. A lokális beállításokhoz hasonló módon adhatunk hozzá, távolíthatunk el és konfigurálhatunk \code{Operation}-öket, a szolgáltatástól kapott naplóüzenetek pedig ugyanott láthatóak. A távoli eszközről a "\emph{Snapshot}" gomb segítségével kérhetünk le kameraképet és az FPS mezőbe beírt értékkel változtathatjuk a kiolvasás sebességét.\\
Lehetőség van a projektek elmentésére és betöltésére is, ez a bal fölső sarokban található "\emph{File}" gombra kattintva érhető el. A projektek kimentése esetén XML fájlba íródik beállított fájlnév, az IP cím és a port mezők tartalma, valamint a képfeldolgozó műveletek a megfelelő konfigurációval. Mentett munkamenet megnyitásakor a projektfájl kiválasztása után a program példányosítja az objektumokat a szerializált formátumból.
 
\begin{figure}[h]
\vspace{.5cm}
\includegraphics[scale=1]{operation-selector.png}
\centering
\vspace{.2cm}
\caption{Művelet választó ablak}
\vspace{.5cm}
\label{fig:operation-selector}
\end{figure}

\begin{figure}[h]
\vspace{.5cm}
\includegraphics[width=1\textwidth]{client-remote.png}
\centering
\vspace{.2cm}
\caption{Távoli eszköz felügyelete}
\vspace{.5cm}
\label{fig:client-remote}
\end{figure}



\section{Architektúra}

A teljes grafikus komponens a \emph{model-view-view model} (MVVM) tervezési minta alapján készült. Az MVVM minta a \emph{model-view-controller}, grafikus felületek tervezésére széleskörűen használt tervezési mintának a specializált változata a WPF platformra -- valójában a teljes keretrendszer ennek a \emph{design pattern}-nek a támogatását szem előtt tartva készült. Az MVVM célja a hasonló tervezési mintákhoz hasonlóan a megjelenített nézet és a prezentált adat közötti csatolás minimálisra csökkentése. A tervezési minta ezt úgy valósítja meg, hogy a felhasználó által látott nézet és a nézetet megtöltő adatmodell közé egy \emph{view model} nevű  komponenst ír elő. A \emph{view model} a nézet absztrakciója, egy közvetítő összetevő, mely nem rendelkezik referenciával a \emph{view}-ra, hanem a nézet objektum kötődik a nézet modell megjelenítés szempontjából fontos \emph{property}-jeire úgynevezett \code{Binding} objektumokon keresztül. Cserébe a nézet modell komponens a modellben tárolt adatokat és egyéb, a megjelenítés szempontjából fontos állapotokat felfedi a nézet számára.\\
A \emph{view} komponens referenciával rendelkezik a \emph{view model}-re, a \emph{view model} pedig a modell(ek)re, azonban ebben az adatmodell teljesen független a nézetmodelltől, a nézetmodell pedig nem tud róla, milyen módon kerül megjelenítésre. Így a komponensek közötti csatolás alacsony szintű, és az egyes komponensek jól tesztelhetők. A \emph{view modell} teszteléséhez ugyanis nincs szükség a valódi grafikus felület megjelenítésére, az egyes funkciókat tesztkódból is hívhatjuk ellenőrizve az elvárt eredményt. Megjegyzendő, hogy a nézetnek nincs tudomása a megjelenített adatmodellről, egyedül a nézetmodell \emph{property}-jein és metódusain keresztül fér hozzá. \cite{wpf-mvvm-pattern}\\


\begin{figure}[h]
\vspace{.5cm}
\includegraphics[width=1\textwidth]{mvvm_pattern.pdf}
\centering
\vspace{.2cm}
\caption{\emph{Model-view-view model} tervezési minta komponensei}
\vspace{.5cm}
\label{fig:mvvm_pattern}
\end{figure}

\section{Nézet}

A WPF keretrendszerben az alkalmazás ablakainak létrehozását a .NET Framework \code{Window} osztályából leszármazó osztályok példányosítását jelenti. Az ablakot reprezentáló osztály tartalmazza a grafikus elemek példányait és a metódusai kezeli a felhasználói interakciókat. Annak érdekében, hogy a felületi elemek és a működésüket meghatározó logika minnél kevésbé függjön egymástól, a bevezető \ref{subs:wpf_intro} pontjában leírt módon, a \emph{view} komponens -- maga az ablak --  statikus és dinamikus részekre osztható.

\subsection{Statikus modell}

A felhasználói felületen megjelenített objektumok elrendezését, tulajdonságait és a hozzájuk tartozó \emph{binding}-okat egy Microsoft által kifejlesztett, XAML (\emph{ Extensible Application Markup Language}) nevű jelölőnyelven írt fájlban adhatjuk meg. A konfigurációs fájlban a grafikus objektumokat XAML \emph{node}-ok reprezentálják, a tulajdonságaikat pedig a megfelelő \emph{node} attribútumok beállítására is lehetőség van. A program indításakor a WPF keretrendszer \emph{markup} fájl alapján példányosítja az objektumokat.\\
Ha az egyes grafikus elemek valamely tulajdonságát a egy nézetmodell tagváltozóhoz kötjük \code{Binding} objektummal szükség lehet valamilyen konverzióra, amely a változók közötti típuskonverziót adja meg. Ilyen esetben lehetőség van megadni egy saját osztályt, amely megvalósítja a \code{IValueConverter} interfészt, a WPF \emph{framework} ennek \code{Convert} és \code{ConvertBack} metódusainak hívásával konvertálja az értékeket oda és vissza. \\
A fájl elemeire egy példa a \ref{fig:client-local} ábrán látható grafikus felületen a lejátszás/szünet gomb objektum XAML definíciója:

\begin{mdframed}[backgroundcolor=gray!20]
\begin{small}
\begin{scriptsize}
\begin{lstlisting}[language=XML]
<Fluent:Button x:Name="PlayPauseButton" 
			   Header="{Binding VideoState, 
			            Converter={RpiConverters:VideoStateConverter}, 
			            ConverterParameter=Play|Pause, 
			            Mode=Default}" 
               LargeIcon="{Binding VideoState,
               			   Converter={RpiConverters:VideoStateConverter}, 
               			   ConverterParameter=play.png|pause.png,
               			   Mode=Default}" 
               Click="Play_Pause_Click" 
               Margin="5"/>
\end{lstlisting}
\end{scriptsize}
\end{small}
\end{mdframed}

A példa egy, a \code{Fluent} névtéren található \code{Button} objektumot reprezentál, amelyre a \emph{view} kódjában \code{PlayPauseButton} névvel hivatkozhatunk. A hozzá tartozó feliratot a videó állapotának aktuális értéke szabja meg, ezt a nézetmodell programkódjában \code{Enum} típusú változó reprezentálja. A \code{VideoStateConverter} osztály egy értékkonverter, amelynek \code{Convert} metódusa a videó \code{Paused} és \code{Stopped} állapota esetén az első  paraméterként adott "\emph{Play}", \code{Playing} állapotában pedig a másik "\emph{Pause}" \code{string} jelenig meg. Az ikon megjelenítésére két képfájl van a Visal Stuido projektben, ezek között a felirat megjelenítéséhez hasonlóan mindig a videólejátszó állapotának megfelelő jelenik meg. \\
A \code{Fluent:Button} kattintására eseménykezelő függvény is beállítható, ha \code{Click} attribútum értékére beállítjuk a megfelelő metódus nevét. Az XAML fájlban a példához hasonló módon vannak definiálva az egyes felületi elemek.

\subsection{Dinamikus modell}

A főablak megvalósításához egy külsős csomagot, a \emph{Fluent Ribbon} \code{RibbonWindow} osztályát használtam, ahogy az összes, \emph{cv4s} keretrendszerre támaszkodó alkalmazás. Az osztály a Microsoft Office termékekben megszokott kezelőfelületet biztosítja, a \code{RibbonWindow} osztályból származik le a főablakot reprezentáló osztály. Ebben az osztályban találhatóak a felhasználói eseményeket kezelő függvények, valamint a konstruktora példányosítja a szükséges erőforrásokat, például a nézetmodellt. A A WPF keretrendszer alkalmazása esetén az nézetmodell átadása a nézetnek igen kézenfekvő módon a \code{Window} objektum \code{DataContext} \emph{property}-jének beállításával történik, a \emph{view modell} változóihoz az XAML fájlban is definiálhatunk \emph{binding}-okat is az MVVM tervezési minta támogatása érdekében.\\
Az implementált osztály fő funkciója a háttérbeli adatobjektumok példányosítása, a működése során feladata kimerül a felhasználói események továbbításában a nézetmodell komponensnek, az előidézett változások eredményei a már létrehozott \code{Binding} objektumokon keresztül jutnak érvényre az ablakban. Az inicializálás során a következő események történnek meg:
\begin{enumerate}
\item \code{InitializeComponents()}
\item \emph{view modell} példányosítása
\item \code{SetUpCanvas()}
\item naplózó objektum példányosítása
\item nézetmodell beállításai
    \begin{enumerate}
   	\item viewModel.Log = logger;
    \item viewModel.LogHistory = new ObservableCollection<Tuple<string, string, string>>();
    \item viewModel.PropertyChanged += ViewModel\_OnPropertyChanged;
    \end{enumerate}
\item \code{SetUpOperations()}
\end{enumerate}

\subsection{Naplózás} 

\section{Nézetmodell}

\section{Modell}

\subsection{Lokális adatmodell}

\subsection{Távoli adatmodell}

\section{Működés}

Az MVVM tervezési minta gyakorlatban való alkalmazására a kliens néhány funkciójához készítettem szekvenciadiagramot is.
