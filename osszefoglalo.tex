\chapter*{Összefoglaló}

A digitális képfeldolgozás a 21. század egyik legszélesebb körben tért nyerő technológiája, amely az ipari gyártósorok minőség-ellenőrzésétől kezdve a forgalomszámláláson át a tömegközlekedési eszközök biztonsági felügyeletéig alakítja mindennapi életünket. A technológiának az 1960-as években letett alapjai után nagy fellendülést hozott a 2000-es évek elején megjelenő, egyre nagyobb számítási kapacitással rendelkező processzorok, valamint a 2010-es évektől egyre nagyobb figyelmet kapó gépi tanulás alapú algoritmusok.\\
\\
Az informatikában rövid története során új megoldások megjelenésekor mindig több eltérő irányba indult el a fejlődés, majd néhány év lefutása alatt a tudományág belső rendező ereje pár rivális technológiát hagyott csupán az utókornak. Ez történt a hardverarchitektúrákkal, az operációs rendszerekkel, de a szoftverfejlesztési szemléletekkel is. A mérnöki gyakorlatban gyakori feladat a különböző, egymással nem kompatibilis rendszerek vagy komponensek összehangolása, esetleg a közülük való választás. Diplomatervemben az eredendően eltérő szemlélet, illetve célkitűzés mellett készült beágyazott környezetben futtatott Linux rendszer, valamint egy .NET \emph{framework}-ön alapuló szoftver -- pontosabban a \emph{cv4s} keretrendszer -- egy alkalmazásba való integrálását végeztem el.\\
\\
A dolgozat egy beágyazott környezetben futó képfeldolgozó modul és a hozzá tartozó felügyeleti kliensalkalmazás megvalósítását és a tervezés során tett megfontolásokat tartalmazza; ezen kívül több, a szoftver implementálásához kapcsolódó technológia is bemutatásra kerül, többek között a \emph{Windows Communication Foundation} és a \emph{Windows Presentation Foundation}.\\
\\
Az elkészült szoftverkomponensek egy olyan, a későbbiekben akár gyakorlatban is használható alkalmazás alapját adják, amellyel valós idejű képfeldolgozást igénylő, általános feladatot beágyazott számítógép segítségével megoldhatunk, és az eszközt távolról felügyelhetjük és konfigurálhatjuk.\\
\\
Az itt leírt megoldások és megfontolások kiindulási alapot jelenthetnek a \emph{cv4s} keretrendszernek az asztali környezettől való elszakadását célzó jövőbeli terveknek.
